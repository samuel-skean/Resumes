%-------------------------
% Resume in Latex
% Author : Jake Gutierrez
% Based off of: https://github.com/sb2nov/resume
% License : MIT
%------------------------

\documentclass[letterpaper,11pt]{article}

\usepackage{latexsym}
\usepackage[empty]{fullpage}
\usepackage{titlesec}
\usepackage{marvosym}
\usepackage[usenames,dvipsnames]{color}
\usepackage{verbatim}
\usepackage{enumitem}
\usepackage[hidelinks]{hyperref}
\usepackage{fancyhdr}
\usepackage[english]{babel}
\usepackage{tabularx}
\usepackage{hyperref}

\urlstyle{same}
\input{glyphtounicode}


%----------FONT OPTIONS----------
% sans-serif
% \usepackage[sfdefault]{FiraSans}
% \usepackage[sfdefault]{roboto}
% \usepackage[sfdefault]{noto-sans}
% \usepackage[default]{sourcesanspro}

% serif
% \usepackage{CormorantGaramond}
% \usepackage{charter}


\pagestyle{fancy}
\fancyhf{} % clear all header and footer fields
\fancyfoot{}
\renewcommand{\headrulewidth}{0pt}
\renewcommand{\footrulewidth}{0pt}

% Adjust margins
\addtolength{\oddsidemargin}{-0.5in}
\addtolength{\evensidemargin}{-0.5in}
\addtolength{\textwidth}{1in}
\addtolength{\topmargin}{-.5in}
\addtolength{\textheight}{1.0in}

\urlstyle{same}

\raggedbottom
\raggedright
\setlength{\tabcolsep}{0in}

% Sections formatting
\titleformat{\section}{
  \vspace{-4pt}\scshape\raggedright\large
}{}{0em}{}[\color{black}\titlerule \vspace{-5pt}]

% Ensure that generate pdf is machine readable/ATS parsable
\pdfgentounicode=1

%-------------------------
% Custom commands
\newcommand{\resumeItem}[1]{
  \item\small{
    {#1 \vspace{-2pt}}
  }
}

\newcommand{\resumeSubheading}[4]{
  \vspace{-2pt}\item
    \begin{tabular*}{0.97\textwidth}[t]{l@{\extracolsep{\fill}}r}
      \textbf{#1} & #2 \\
      \textit{\small#3} & \textit{\small #4} \\
    \end{tabular*}\vspace{-7pt}
}

\newcommand{\resumeSubSubheading}[2]{
    \item
    \begin{tabular*}{0.97\textwidth}{l@{\extracolsep{\fill}}r}
      \textit{\small#1} & \textit{\small #2} \\
    \end{tabular*}\vspace{-7pt}
}

\newcommand{\resumeProjectHeading}[2]{
    \item
    \begin{tabular*}{0.97\textwidth}{l@{\extracolsep{\fill}}r}
      \small#1 & #2 \\
    \end{tabular*}\vspace{-7pt}
}

\newcommand{\resumeSubItem}[1]{\resumeItem{#1}\vspace{-4pt}}

\renewcommand\labelitemii{$\vcenter{\hbox{\tiny$\bullet$}}$}

\newcommand{\resumeSubHeadingListStart}{\begin{itemize}[leftmargin=0.15in, label={}]}
\newcommand{\resumeSubHeadingListEnd}{\end{itemize}}
\newcommand{\resumeItemListStart}{\begin{itemize}}
\newcommand{\resumeItemListEnd}{\end{itemize}\vspace{-5pt}}

%-------------------------------------------
%%%%%%  RESUME STARTS HERE  %%%%%%%%%%%%%%%%%%%%%%%%%%%%


\begin{document}

%----------HEADING----------
% \begin{tabular*}{\textwidth}{l@{\extracolsep{\fill}}r}
%   \textbf{\href{http://sourabhbajaj.com/}{\Large Sourabh Bajaj}} & Email : \href{mailto:sourabh@sourabhbajaj.com}{sourabh@sourabhbajaj.com}\\
%   \href{http://sourabhbajaj.com/}{http://www.sourabhbajaj.com} & Mobile : +1-123-456-7890 \\
% \end{tabular*}

\begin{center}
    \textbf{\Huge \scshape Samuel Skean} \\ \vspace{1pt} \href{mailto:skeansamuel64@gmail.com}{\underline{skeansamuel64@gmail.com}} $|$ 
    LinkedIn: \href{https://www.linkedin.com/in/samuel-skean-nod/}{\underline{samuel-skean-nod}} $|$ \href{https://samuel-skean.github.io}{\underline{samuel-skean.github.io}}
\end{center}


%-----------EDUCATION-----------
\section{Education}
  \resumeSubHeadingListStart
    \resumeSubheading
      {University of Illinois at Chicago}{Chicago, IL}
      {Master's of Science in Computer Science}{Aug. 2024 -- Expected May 2026}
      \begin{itemize}
        \item \textbf{GPA}{: Not Yet Known}
        \item \textbf{Relevant Coursework}: Operating Systems, Compilers, Cloud Computing.
      \end{itemize}
    \resumeSubheading
      {University of Illinois at Chicago}{Chicago, IL}
      {Bachelor of Science in Computer Science}{Aug. 2020 -- May 2024}
      \begin{itemize}
        \item \textbf{GPA}{: 3.93/4.0}
        \item \textbf{Relevant Coursework}: Web App Development, Data Structures, Systems Programming, Programming Language Design, Principles of Concurrent Computing, Systems Performance and Concurrent Computing, Graphics.
      \end{itemize}
    % \resumeSubheading
    %   {Wheaton Warrenville South High School}{Wheaton, IL}
    %   {Graduated While Taking Only AP and Honors Level Classes}{Aug. 2017 -- May 2020}
    %   \begin{itemize}
    %     \item \textbf{GPA: 3.9/4.0, ACT: 36, SAT: 1530}
    %     \item Relevant College Level Classes: AP Computer Science A, AP Calculus BC, AP Physics C: Electricity And Magnetism.
    %   \end{itemize}
  \resumeSubHeadingListEnd

%
%-----------PROGRAMMING SKILLS-----------
\section{Technical Skills}
 \begin{itemize}[leftmargin=0.15in, label={}]
    \small{\item{
     \textbf{Languages}{: C/C++, Rust, Java, C\verb|#|, Dart, Swift, Python, SQL (SQLite), JavaScript, HTML/CSS, F\verb|#|, OCaml, Matlab, Bash, AWK, x86 Assembly (AT\verb|&|T)} \\
     \textbf{Frameworks/Libraries}{: JavaFX, React.js, Flutter, Axum (web framework), Matplotlib, WebGL2, p5.js} \\
    }}
 \end{itemize}
 
%-----------EXPERIENCE-----------
\section{Experience}
  \resumeSubHeadingListStart
    \resumeSubheading
      {CS Teaching Assistant (Undergrad and Grad)}{January 2023 -- Present}
      {UIC}{Chicago, IL}
      \resumeItemListStart
        \resumeItem{Helped students with syntax and Data Structures (lists, trees, hashmaps) in C++}
        \resumeItem{Helped students understand database, functional, and concurrent programming in a Survey Course}
        \resumeItem{Helps debug simple embedded projects in C on Arduino}
        \resumeItem{Proctors exams and labs, giving short lessons on related topics}
      \resumeItemListEnd
    \resumeSubheading
      {Student Ambassador for National Science Foundation Engineering Scholarship}{August 2024}
      {UIC}{Chicago, IL}
      \resumeItemListStart
        \resumeItem{Taught a short, custom lesson on algorithmic thinking, and helped with lessons on logic gates}
        \resumeItem{Offered advice on classes, professors, and skills relevant to CS and engineering}
    \resumeItemListEnd
    \resumeSubheading
      {Information Technology Support Specialist}{August 2021 -- December 2022}
      {UIC Technology Solutions}{Chicago, IL}
      \resumeItemListStart
        \resumeItem{Communicated with managers to help with everything from enrollment to software troubleshooting}
        \resumeItem{Demonstrated patience with older/technology-unfamiliar people and those in stressful, unfamiliar situations}
        \resumeItem{Troubleshooted new services and software packages daily, keeping track of overlapping systems and the credentials they take}
      \resumeItemListEnd
      
% -----------Multiple Positions Heading-----------

    % \resumeSubheading
    %   {Frontend Sales Associate At Grocery Store}{June 2019 -- August 2020}
    %   {Mariano's Fresh Market}{Wheaton, IL}
    %   \resumeItemListStart
    %     \resumeItem{Presented a friendly face to customers when bagging groceries and pushing carts}
    %     \resumeItem{Helped customers with occasional questions and issues with the store}
    %   \resumeItemListEnd

  \resumeSubHeadingListEnd


%-----------PROJECTS-----------
\section{Projects}
    \resumeSubHeadingListStart
        \resumeProjectHeading
          {\textbf{\href{https://github.com/samuel-skean/My_Raytracing_Adventures.git}{\underline{Path Tracer and Bezier Drawer}}} $|$ \emph{Rust, SDL2, pixels, winit, serde}}{February 2024 -- August 2024}
          \resumeItemListStart
            \resumeItem{Developed a simple path tracer (a kind of 3D renderer), mostly following Raytracing in One Weekend by Peter Shirley et al.}
            \resumeItem{Used JSON to create a format to represent the world, and randomly generate spheres within that world}
            \resumeItem{Added a command-line frontend and a concurrent graphical preview of the rendering}
            \resumeItem{Wrote a tool to draw bezier curves and splines, with a simple GUI}
          \resumeItemListEnd
        % \resumeProjectHeading
        %   {\textbf{15-Puzzle Graphical Game} $|$ \emph{Java, JavaFX}}{November 2021}
        %   \resumeItemListStart
        %     \resumeItem{Developed a GUI application to let the player solve a 15-puzzle, a puzzle where numbers must be arranged in a certain way in a grid}
        %     \resumeItem{Used A* search to solve the puzzle if the player asks}
        %     \resumeItem{Implemented asynchronous UI and worker threads to keep the app responsive while the puzzle-solving code was busy}
        %   \resumeItemListEnd
          
%      \resumeProjectHeading
%          {\textbf{} $|$ \emph{F#}}{April 2022}
%          \resumeItemListStart
%            \resumeItem{Used F# to interpret the code in a file according to Backus-Naur form in entirely functional paradigm, with no side-effects, for safely extensible code}
%          \resumeItemListEnd
      \resumeProjectHeading
          {\textbf{\underline{Tracing Garbage Collector}} $|$ \emph{C}}{December 2022}
          \resumeItemListStart
            \resumeItem{Implemented a mark-and-sweep garbage collector in C}
            \resumeItem{Created a simple memory allocator using sbrk()}
            \resumeItem{Manipulated pointers to find all allocated, unused memory on the heap and free it without the user calling free}
          \resumeItemListEnd
      % \resumeProjectHeading
      %     {\textbf{Nullability in a Simple OO Programming Language} $|$ \emph{OCaml}}{December 2023}
      %     \resumeItemListStart
      %       \resumeItem{Added the feature of nullability as seen in Swift, Dart, and Kotlin to a simple object-oriented programming language}
      %       \resumeItem{Statically ensured programs could not crash by dereferencing null}
      %       \resumeItem{Implemented typechecker and interpreter in OCaml in pure functional style}
      %     \resumeItemListEnd
    \resumeSubHeadingListEnd


%-------------------------------------------
\end{document}
